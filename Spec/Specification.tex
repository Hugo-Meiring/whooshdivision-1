\documentclass[a4paper,12pt]{report}
\addtolength{\oddsidemargin}{-1.cm}
\addtolength{\textwidth}{2cm}
\addtolength{\topmargin}{-2cm}
\addtolength{\textheight}{3.5cm}
\newcommand{\HRule}{\rule{\linewidth}{0.5mm}}
\makeindex

\usepackage{longtable}
\usepackage{graphicx}
\usepackage{makeidx}
\usepackage{hyperref}
\usepackage{verbatim}

\hypersetup{
    colorlinks=true,
    linkcolor=black,
    filecolor=magenta,      
    urlcolor=cyan,
}


% define the title
\author{Whoosh Division}
\title{OcuViz - EpiUse Labs}
\begin{document}
\setlength{\parskip}{6pt}

% generates the title
\begin{titlepage}

\begin{center}
% Upper part of the page       
\includegraphics[width=1\textwidth]{./images/up-logo.jpg}\\[0.4cm]    
\textsc{\LARGE Department of Computer Science}\\[1.5cm]
\textsc{\Large Whoosh Division}\\[0.5cm]
% Title
\HRule \\[0.4cm]
{ \huge \bfseries OcuViz - EpiUse Labs}\\[0.4cm]
\HRule \\[0.4cm]
\begin{minipage}{0.4\textwidth}
\begin{flushleft} \large
Vukile {Langa}
\end{flushleft}
\end{minipage}
\begin{minipage}{0.4\textwidth}
\begin{flushright} \large
\emph{} \\
u14035449 
\end{flushright}
\end{minipage}
\begin{minipage}{0.4\textwidth}
\begin{flushleft} \large
Wynand Hugo Meiring
\end{flushleft}
\end{minipage}
\begin{minipage}{0.4\textwidth}
\begin{flushright} \large
\emph{} \\
u13230795  
\end{flushright}
\end{minipage}
\begin{minipage}{0.4\textwidth}
\begin{flushleft} \large
Nontokozo Hlastwayo
\end{flushleft}
\end{minipage}
\begin{minipage}{0.4\textwidth}
\begin{flushright} \large
\emph{} \\
u14414555
\end{flushright}
\end{minipage}
\begin{minipage}{0.4\textwidth}
\begin{flushleft} \large
Gerome Schutte
\end{flushleft}
\end{minipage}
\begin{minipage}{0.4\textwidth}
\begin{flushright} \large
\emph{} \\
u12031519
\end{flushright}
\end{minipage}
\vfill

{\large \today}
\end{center}
\end{titlepage}
\footnotesize
\normalsize

%
%	Table of Contents
%

\tableofcontents
\newpage

%
%	Document
%
%	The detailed functional requirements around individual use cases will be done iteratively in the
% 	context of agile software development, i.e. the subsequent documentation submissions will include the
% 	detailed functional requirements, application design, user documentation and test documentation for
%	those use cases which have been completed since the last demo.
%

\section{Vision}

	% To-do
	The vision of this project is to enable everyday people to use the power of virtual reality to grasp a sense of scale in manners that have usually been misunderstood. This would allow easy and feasible comparisons, compared to their real-world counterparts.

\newpage
%
%	End of Vision
%




\section{Project Scope}		% (see scope specification for mini-project)

	% To-do
	A user can select a scenario they would like to visualise using Oculus Rift. Users can also specify their desired scene using inputs which will be generated and rendered for the user after objects are collected from a cloud store.

\newpage
%
%	End of Project Scope
%


\section{Architectural Requirements}	% Non-functional Requirements

\subsection{Access and integration requirements}

\subsection{Quality requirements}
Code needs to follow high development standards to produce professional code. Source code needs to be easily understood and maintainable.

\subsubsection{Flexibility}
The system needs to be able to accept a range of different object / model file types for both 2D and 3D files. CSV as well as other input data formats need to be supported.

\subsubsection{Maintainability}
	\begin{itemize}
		\item The system should be modular and allow easy updating and fixing in the future.
	\end{itemize}

\subsubsection{Scalability}
All major platforms should be catered for in support or aleast be relatively simple to port to. This includes:
	\begin{itemize}
		\item Windows
		\item Mac
		\item Linux		
	\end{itemize}
An added bonus would be:
	\begin{itemize}
		\item Android
		\item iOS
		\item Any other mobile platform (Windows mobile, etc)
	\end{itemize}
	
\subsubsection{Performance requirements}
To ensure the best experience and prevent any motion sequence meeting performance requirements is crucial.
	\begin{itemize}
		\item Consistent and high frame rate (75 or more fps or 13.33 ms max per frame)
		\item Low latency between input and display
	\end{itemize}
Oculus Rift's requirements for CV1 (Consumer Version 1) has the following explicitely stated requirements:
\begin{itemize}
	\item NVIDIA GTX 970 / AMD 290 equivalent or greater.
	\item Intel i5-4590 equivalent or greater.
	\item 8GB+ RAM.
	\item Compatible HDMI 1.3 video output.
	\item 2x USB 3.0 ports.
	\item Windows 7 SP1 or newer.
\end{itemize}
Working backwards, DK2 requires atleast a GTX 770 equivelant card or better owning to the fact that it has 25\% less pixels compared to the CV1. Also it only needs to run at 75 fps compared to 90 fps of the CV1.
\subsubsection{Reliability}
	\begin{itemize}
		\item The system should be able to handle user input without crashing.
	\end{itemize}

\subsubsection{Security}
Currently security is not part of the requirements, however this could change at any stage.

\subsubsection{Auditability}

\subsubsection{Testability}
There should be proper unit tests which cover all the contracts of the project. All tests should be automated and allow for both mock objects and integration tests.

\subsubsection{Usability}
	\begin{itemize}
		\item The system must have a simple and easy to use interface
		\item The system should not require training before use
		\item Designing scenarios should be simple
	\end{itemize}

\subsubsection{Integrability}

\subsubsection{Deployability}

\subsection{Architectural responsibilities}

\subsection{Architecture constraints}

\section{Architecture design?}
	Is this needed?? Subsections include: architectural tactics, architectural components addressing architectural responsibilities, infrastructure, concepts and contraints for application components

\newpage
%
%	End of Architectural Requirements
%




\section{Initial Design}  % proposed software architecture (this may evolve over the project)

	% To-do

\newpage
%
%	End of Initial Design
%
\end{document}
