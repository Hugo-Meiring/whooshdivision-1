\documentclass[a4paper,12pt]{article}
\addtolength{\oddsidemargin}{-1.cm}
\addtolength{\textwidth}{2cm}
\addtolength{\topmargin}{-2cm}
\addtolength{\textheight}{3.5cm}
\newcommand{\HRule}{\rule{\linewidth}{0.5mm}}
\makeindex

\usepackage{longtable}
\usepackage{graphicx}
\usepackage{makeidx}
\usepackage{hyperref}
\usepackage{verbatim}

\hypersetup{
    colorlinks=true,
    linkcolor=black,
    filecolor=magenta,      
    urlcolor=cyan,
}


% define the title
\author{Whoosh Division}
\title{OcuViz - EpiUse Labs}
\begin{document}
\setlength{\parskip}{6pt}

% generates the title
\begin{titlepage}

\begin{center}
% Upper part of the page       
\includegraphics[width=1\textwidth]{./images/up-logo.jpg}\\[0.4cm]    
\textsc{\LARGE Department of Computer Science}\\[1.5cm]
\textsc{\Large Whoosh Division}\\[0.5cm]
% Title
\HRule \\[0.4cm]
{ \huge \bfseries OcuViz - EpiUse Labs}\\[0.4cm]
\HRule \\[0.4cm]
\begin{minipage}{0.4\textwidth}
\begin{flushleft} \large
Vukile {Langa}
\end{flushleft}
\end{minipage}
\begin{minipage}{0.4\textwidth}
\begin{flushright} \large
\emph{} \\
u14035449 
\end{flushright}
\end{minipage}
\begin{minipage}{0.4\textwidth}
\begin{flushleft} \large
Wynand Hugo Meiring
\end{flushleft}
\end{minipage}
\begin{minipage}{0.4\textwidth}
\begin{flushright} \large
\emph{} \\
u13230795  
\end{flushright}
\end{minipage}
\begin{minipage}{0.4\textwidth}
\begin{flushleft} \large
Nontokozo Hlatshwayo
\end{flushleft}
\end{minipage}
\begin{minipage}{0.4\textwidth}
\begin{flushright} \large
\emph{} \\
u14414555
\end{flushright}
\end{minipage}
\begin{minipage}{0.4\textwidth}
\begin{flushleft} \large
Gerome Schutte
\end{flushleft}
\end{minipage}
\begin{minipage}{0.4\textwidth}
\begin{flushright} \large
\emph{} \\
u12031519
\end{flushright}
\end{minipage}
\vfill

{\large \today}
\end{center}
\end{titlepage}
\footnotesize
\normalsize

%
%	Table of Contents
%

\tableofcontents
\newpage

%
%	Document
%
%	The detailed functional requirements around individual use cases will be done iteratively in the
% 	context of agile software development, i.e. the subsequent documentation submissions will include the
% 	detailed functional requirements, application design, user documentation and test documentation for
%	those use cases which have been completed since the last demo.
%

\section{Vision}

	% To-do
	The vision of this project is to enable everyday people to use the power of virtual reality to grasp a sense of scale in manners that have usually been misunderstood. This would allow easy and feasible comparisons, compared to their real-world counterparts. Furthermore, this system would need to make the process of user specified model as efficient as possible, while maintaining high performance. 
	\newline\newline
	Epi-Use Labs specialises in creating unique solutions to issues in industry, including the business environment. Through the power of OcuViz, Epi-Use Labs will be able to visualise their experiments and research spatial relationships of entities.
	\newline
	Above being used by Epi-Use Labs, this could be used in the education and architectural environments, as the visualisation would easily allow average users to compare spatial relationships, and be part of a scene. Other users can also specify their own defined scenes in combination of an editor and object files.
	\newline
	This project is not intended for professionals in the gaming or graphics industry, who expect a fully functioning game engine to render their scenes. This project is intended specifically for the users who want to visualise statistics and spatial relationships that would otherwise be infeasible or impossible in real life.
%
%	End of Vision
%




\section{Project Scope}		% (see scope specification for mini-project)

	% To-do
	The main purpose of the project is to enable data visualisation with the aid of virtual reality. Casual users will be able to select rather complex, pre-set scenarios to be visualised using the Oculus Rift Virtual Reality (VR) Headset, and the software will also allow generation of new scenarios from input data and 3D models which may be imported from a supported cloud store. Recently generated scenes are cached locally, and sometimes recycled to enhance efficiency and reduce traffic between requesting a scene and viewing one.
%
%	End of Project Scope
%
\newpage
\section{Functional Requirements}
\subsection{Allow the user to create scenarios:}
-The user should upload two csv files description file and input file . Where they specify what objects should be in the scene, how big the object should, if there should be animation in the scene, what type of animation(the speed, direction distance and other details about animation) and to which objects should the animation be applied and the scale of the viewer whether it should be fixec or scalable.Also if a user want to import models or scene they should be able or if they to download the models or scene they should be able to do so.There must be a editor to load and save the scene description file.
-Data format : The system must support 2D and 3D input data.Rows in the input data will be linked to entities. Entities will have properties(like the ssize of the model) and this will be linked to either static values or columns in the input data.
-Scene format :Specifying placement of entities in 3D(rows, randomly, specified by data or globe/map based on location information).Background shoud be a solid colour and/or cubemap.
Placing entities on globe/map based on location info. 
-Models: Each entity will have an associated model. Basic shapes the will be supported(sphere, cylinder and rectangular prism). Texturing( and colouring will also be supported. Import of various types of models will be supported (Objects will be supported).
\subsection{View scenarios using the Oculus Rift:}
The user will be able to move around the world and they also should be able to use various controls. The users  allow to change their size can either become bigger or smaller, this ratio to an average human will be displayed if the user or the scenario changed the user's relative size. The user will be able to change the rate atwhich time passes, the speed at which time passes will be displayed if the user or scenario has changed the speed at which time passes.


\newpage
\section{Architectural Requirements}	% Non-functional Requirements

\subsection{Access and integration requirements}
As with the technology being used, access would need to be using an Oculus Rift. \newline\newline The application would need to be run from a supported computer. This computer would need to be a major platform, such as, but not limited to
\begin{itemize}
	\item Windows
\end{itemize}
Other platforms that would be ideal, but do not have support for Rift yet include
\begin{itemize}
	\item Mac
	\item Linux
\end{itemize}
These will be extended to the access requirements once Oculus has support for these platforms.

\subsection{Quality requirements}
Code needs to follow high development standards to produce professional code. Source code needs to be easily understood and maintainable.

\subsubsection{Flexibility}
\begin{enumerate}
	\item \textbf{File type support:}\\
	The system needs to support a range of different file types for both 2D and 3D files and input data:
	\begin{itemize}
		\item \textbf{.csv:} For data input of describing the scene and input files for entities, where each row represents a single entity.
		\item \textbf{.obj:} For object models. Various model stores provide models in this format, and many 3D modelling applications are able to export models in this format.
		\item \textbf{.fbx:} For other object models. These models would be imported from the SketchUp library when a user specifies them from an input file.
	\end{itemize}
\end{enumerate}


\subsubsection{Maintainability}
	\begin{itemize}
		\item The system should be modular and allow easy updating and fixing in the future.
	\end{itemize}

\subsubsection{Scalability}
All major platforms should be catered for in support or at least be relatively simple to port to. This includes:
	\begin{itemize}
		\item Windows
		\item Mac
		\item Linux		
	\end{itemize}
An added bonus would be:
	\begin{itemize}
		\item Android
		\item iOS
		\item Any other mobile platform (Windows mobile, etc.)
	\end{itemize}
	
\subsubsection{Performance requirements}
To ensure the best experience and prevent any motion sickness meeting performance requirements is crucial.
	\begin{itemize}
		\item Consistent and high frame rate (75 or more fps or 13.33 ms max per frame)
		\item Low latency between input and display
	\end{itemize}
Oculus Rift's requirements for CV1 (Consumer Version 1) has the following explicitly stated requirements:
\begin{itemize}
	\item NVIDIA GTX 970 / AMD 290 equivalent or greater.
	\item Intel i5-4590 equivalent or greater.
	\item 8GB+ RAM.
	\item Compatible HDMI 1.3 video output.
	\item 2x USB 3.0 ports.
	\item Windows 7 SP1 or newer.
\end{itemize}
Working backwards, DK2 requires at least a GTX 770 equivalent card or better owning to the fact that it has 25\% less pixels compared to the CV1. Also it only needs to run at 75 fps compared to 90 fps of the CV1.
\subsubsection{Reliability}
	\begin{itemize}
		\item The system should be able to handle user input without crashing.
	\end{itemize}

\subsubsection{Security}
Currently security is not part of the requirements, however this could change at any stage.

%\subsubsection{Auditability}


\subsubsection{Testability}
There should be proper unit tests which cover all the contracts of the project. All tests should be automated and allow for both mock objects and integration tests.

\subsubsection{Usability}
	\begin{itemize}
		\item The system must have a simple and easy to use interface
		\item The system should not require training before use
		\item Designing scenarios should be simple
	\end{itemize}

%\subsubsection{Integrability}

%\subsubsection{Deployability}

%\subsection{Architectural responsibilities}

%\subsection{Architecture constraints}

%\section{Architecture design?}
%	Is this needed?? Subsections include: architectural tactics, architectural components addressing architectural responsibilities, infrastructure, concepts and constraints for application components


%
%	End of Architectural Requirements
%



\newpage
\section{Initial Design}  % proposed software architecture (this may evolve over the project)

\subsection{Microservices Architecture}
The application will have a hexagonal architecture with the four following components.

\subsubsection{Presentation components}
CSV files will be used to receive data from the user which will be the description of the scene, the preferred format of the output and other information required from the user. From this data, necessary input files required by the scene descriptor will be retrieved and processed into a tree for use with rendering the required scene.

\subsubsection{Business logic}
Taking the user's input and processing it to a correct scene.

\subsubsection{Database access logic}
We will need to access the data from Wolfram Alpha and SketchUp to create the scene, as well as Unreal Blueprints. These would be cached locally onto an optimised CSV file containing all entities in the scene. Only the most recent scenes would be cached. Models will also be temporarily cached locally to reduce traffic and increase efficiency.

\subsubsection{Application integration logic}
Output of the whole application is to create the scene description given by the user. The editor allows a user to further specify their own scenes.


\subsection{Technologies}
\subsubsection{Software}
\begin{enumerate}
	\item \textbf{Game engine:}\\
		Developing for VR, the obvious choice was to use a game engine as the underlying technology. Game engines provide a complete solution to many challenges like Virtual Reality support, object modelling, scene rendering, texturing and model file type support, and because of this the game engine itself will form the biggest part of the project's technology stack. The game engine chosen to this end is Unreal Engine 4, as it provides:
		\begin{itemize}
			\item Cross platform support for Windows, Linux and Mac OS.
			\item Built-in support for the Oculus Rift
			\item Support for all general 3D modelling functionality, such as object modelling, scene rendering, texturing, etc.
		\end{itemize}
		\textbf{Alternatives considered:}
		\begin{enumerate}
			\item \textbf{Unity 5:}  This engine comes with a limited free edition, with licenses required for the full version, whereas Unreal is based on a royalty model and otherwise free.%to complete, why choose unreal rather than unity?
			\item \textbf{CryEngine:} %to complete, why choose unreal rather than cryengine?
		\end{enumerate}
\end{enumerate}
\subsubsection{Hardware}
\begin{enumerate}
	\item \textbf{Oculus Rift VR Headset:}\\
		The Oculus Rift is listed as a requirement by the client, and provides integrated support for Unreal Engine 4\\.
		\textbf{Alternatives considered:}
		\begin{itemize}
			\item Since the Oculus Rift was specified as a requirement by the client, no alternatives were considered.
		\end{itemize}
\end{enumerate}
\newpage
%
%	End of Initial Design
%
\end{document}
