\documentclass[a4paper,12pt]{article}
\addtolength{\oddsidemargin}{-1.cm}
\addtolength{\textwidth}{2cm}
\addtolength{\topmargin}{-2cm}
\addtolength{\textheight}{3.5cm}
\newcommand{\HRule}{\rule{\linewidth}{0.5mm}}
\makeindex

\usepackage{longtable}
\usepackage{graphicx}
\usepackage{makeidx}
\usepackage{hyperref}
\usepackage{verbatim}

\hypersetup{
    colorlinks=true,
    linkcolor=black,
    filecolor=magenta,      
    urlcolor=cyan,
}


% define the title
\author{Whoosh Division}
\title{OcuViz - EpiUse Labs}
\begin{document}
\setlength{\parskip}{6pt}

% generates the title
\begin{titlepage}

\begin{center}
% Upper part of the page       
\includegraphics[width=1\textwidth]{./images/up-logo.jpg}\\[0.4cm]    
\textsc{\LARGE Department of Computer Science}\\[1.5cm]
\textsc{\Large Whoosh Division}\\[0.3cm]
% Title
\HRule \\[0.4cm]
{ \huge \bfseries OcuViz - EpiUse Labs}\\[0.4cm]
{ \huge Software Requirements Specification\\[0.4cm]and Technology Neutral Process Design }\\[0.4cm]
\HRule \\[0.4cm]
\begin{minipage}{0.4\textwidth}
\begin{flushleft} \large
Vukile {Langa}
\end{flushleft}
\end{minipage}
\begin{minipage}{0.4\textwidth}
\begin{flushright} \large
\emph{} \\
u14035449 
\end{flushright}
\end{minipage}
\begin{minipage}{0.4\textwidth}
\begin{flushleft} \large
Wynand Hugo Meiring
\end{flushleft}
\end{minipage}
\begin{minipage}{0.4\textwidth}
\begin{flushright} \large
\emph{} \\
u13230795  
\end{flushright}
\end{minipage}
\begin{minipage}{0.4\textwidth}
\begin{flushleft} \large
Nontokozo Hlastwayo
\end{flushleft}
\end{minipage}
\begin{minipage}{0.4\textwidth}
\begin{flushright} \large
\emph{} \\
u14414555
\end{flushright}
\end{minipage}
\begin{minipage}{0.4\textwidth}
\begin{flushleft} \large
Gerome Schutte
\end{flushleft}
\end{minipage}
\begin{minipage}{0.4\textwidth}
\begin{flushright} \large
\emph{} \\
u12031519
\end{flushright}
\end{minipage}
\vfill

{\large \today}
\end{center}
\end{titlepage}
\footnotesize
\normalsize

%
%	Table of Contents
%

\tableofcontents
\newpage

%
%	Document
%
%	The detailed functional requirements around individual use cases will be done iteratively in the
% 	context of agile software development, i.e. the subsequent documentation submissions will include the
% 	detailed functional requirements, application design, user documentation and test documentation for
%	those use cases which have been completed since the last demo.
%

\section{Vision}

	% To-do
	OcuViz is a platform for creating rich visual representations of otherwise unintuitive data by using the power of virtual reality visualisation. It aims to address the time consuming task of working with 3D scenes and make it simple, by providing a format for specifying scenes that is concise, optimised for integration with pluggable data, and to be targeted at being easily usable by anyone with even the slightest development experience.\\\\
	OcuViz would typically be deployed in research, data-auditing and demonstration environments, where visualisations may be used as a medium to bring data to life, for example, describing a scene that gives viewers an interactive walk-around view of a solar cataclysm rather than simply listing numbers representing objects and scale, or allowing viewers to encounter the world from the point of view of a tiny animal, or giving demonstrators the power to transport an audience to a previously unseen location.
	
\section{Background}

	EPI-Use Labs provides products aimed at getting the most out of data. The business prides itself in deploying products which allow users to sync, manipulate, extract and report various forms of data, whether through on-site or cloud-based solutions. The OcuViz project is a natural evolution of this focus. OcuViz enriches the data-reporting process for in-house developers at EPI-Use Labs and business clients alike, by adding new visual and interactive dimensions throughout their project life-cycles, whether used in product development as a tool to interact with test and debugging data, or as a value added feature in a complete solution.

\newpage

\section{Architecture Requirements}

\subsection{Access Channel Requirements}
	
	The system should be accessible via a desktop client, with mobile clients specified as optional, but most likely not feasible due to the hardware requirements of graphics processing. The target platform is Windows x64, and as such a Windows x64 client should be provided in the form of a .exe file, complete with .exe install file.

\subsection{Quality Requirements}
	
	\subsubsection{Performance}
	
		Stringent performance requirements are necessary, not only for novelty purposes, but due to the introduction of virtual reality. In order for visualisations running in virtual reality not to cause motion sickness in the viewer, the following minimum requirements need to be met:
		
		\begin{itemize}
			\item Visualisations produced must run at a constant framerate of 75fps or more.
			\item Overall system latency between user input and display must be kept below a maximum of 20ms.
		\end{itemize}
	
	\subsubsection{Reliability}
	
		\begin{itemize}
			\item Since input data and scene descriptor files may optionally in large part be user- or externally generated, faulty input data and scene descriptor files must be detected and reported without system crash.
			\item All expected user inputs for each visualisation must be clearly defined. Where an input is not declared as "expected", it must have no effect.
			\item The results of rendering a visualisation from a scene descriptor and input data file must be predictable and repeatable. To this end, a complete manual must be provided describing the valid format and possible inputs of a scene descriptor file. 
			\item Objects in a scene created via the scene editor must have the same properties in the actual visualisation as specified the scene editor.
		\end{itemize}
		
	\subsubsection{Scalability}
	
		\begin{itemize}
			\item The system must be developed using technologies which are operating system neutral. Windows x64 is specified as the priority target platform, but the client may in future choose to port the system to another operating system, and must be able to do so without having to rewrite system components.
		\end{itemize}
\section{Architectural Requirements}	% Non-functional Requirements

\subsection{Access and integration requirements}

\subsection{Quality requirements}
Code needs to follow high development standards to produce professional code. Source code needs to be easily understood and maintainable.

\subsubsection{Flexibility}
The system needs to be able to accept a range of different object / model file types for both 2D and 3D files. CSV as well as other input data formats need to be supported.

\subsubsection{Maintainability}
	\begin{itemize}
		\item The system should be modular and allow easy updating and fixing in the future.
	\end{itemize}

\subsubsection{Scalability}
All major platforms should be catered for in support or aleast be relatively simple to port to. This includes:
	\begin{itemize}
		\item Windows
		\item Mac
		\item Linux		
	\end{itemize}
An added bonus would be:
	\begin{itemize}
		\item Android
		\item iOS
		\item Any other mobile platform (Windows mobile, etc)
	\end{itemize}
	
\subsubsection{Performance requirements}
To ensure the best experience and prevent any motion sequence meeting performance requirements is crucial.
	\begin{itemize}
		\item Consistent and high frame rate (75 or more fps or 13.33 ms max per frame)
		\item Low latency between input and display
	\end{itemize}
Oculus Rift's requirements for CV1 (Consumer Version 1) has the following explicitely stated requirements:
\begin{itemize}
	\item NVIDIA GTX 970 / AMD 290 equivalent or greater.
	\item Intel i5-4590 equivalent or greater.
	\item 8GB+ RAM.
	\item Compatible HDMI 1.3 video output.
	\item 2x USB 3.0 ports.
	\item Windows 7 SP1 or newer.
\end{itemize}
Working backwards, DK2 requires atleast a GTX 770 equivelant card or better owning to the fact that it has 25\% less pixels compared to the CV1. Also it only needs to run at 75 fps compared to 90 fps of the CV1.
\subsubsection{Reliability}
	\begin{itemize}
		\item The system should be able to handle user input without crashing.
	\end{itemize}

\subsubsection{Security}
Currently security is not part of the requirements, however this could change at any stage.

\subsubsection{Auditability}

\subsubsection{Testability}
There should be proper unit tests which cover all the contracts of the project. All tests should be automated and allow for both mock objects and integration tests.

\subsubsection{Usability}
	\begin{itemize}
		\item The system must have a simple and easy to use interface
		\item The system should not require training before use
		\item Designing scenarios should be simple
	\end{itemize}

\subsubsection{Integrability}

\subsubsection{Deployability}

\subsection{Architectural responsibilities}

\subsection{Architecture constraints}

\section{Architecture design?}
	Is this needed?? Subsections include: architectural tactics, architectural components addressing architectural responsibilities, infrastructure, concepts and contraints for application components

\newpage
%
%	End of Architectural Requirements
%




\section{Initial Design}  % proposed software architecture (this may evolve over the project)

	% To-do

\newpage
%
%	End of Initial Design
%
\end{document}
