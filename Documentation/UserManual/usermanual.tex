\documentclass[a4paper,12pt]{article}
\addtolength{\oddsidemargin}{-1.cm}
\addtolength{\textwidth}{2cm}
\addtolength{\topmargin}{-2cm}
\addtolength{\textheight}{3.5cm}
\newcommand{\HRule}{\rule{\linewidth}{0.5mm}}
\newcommand\tab[1][1cm]{\hspace*{#1}}
\makeindex

\usepackage{longtable}
\usepackage{graphicx}
\usepackage{makeidx}
\usepackage{hyperref}
\usepackage{verbatim}

\hypersetup{
    colorlinks=true,
    linkcolor=black,
    filecolor=magenta,      
    urlcolor=cyan,
}


% define the title
\author{Whoosh Division}
\title{OcuViz - EpiUse Labs}
\begin{document}
\setlength{\parskip}{6pt}

% generates the title
\begin{titlepage}
\begin{center}
% Upper part of the page       
\includegraphics[scale=1]{../Spec/images/up-logo.jpg}\\[0.4cm]    
\textsc{\LARGE Department of Computer Science}\\[1.5cm]
\textsc{\Large Whoosh Division}\\[0.5cm]
% Title
\HRule \\[0.4cm]
{ \huge \bfseries OcuViz - EpiUse Labs}\\[0.4cm]
\HRule \\[0.4cm]
\begin{minipage}{0.4\textwidth}
\begin{flushleft} \large
Vukile {Langa}
\end{flushleft}
\end{minipage}
\begin{minipage}{0.4\textwidth}
\begin{flushright} \large
\emph{} \\
u14035449 
\end{flushright}
\end{minipage}
\begin{minipage}{0.4\textwidth}
\begin{flushleft} \large
Wynand Hugo Meiring
\end{flushleft}
\end{minipage}
\begin{minipage}{0.4\textwidth}
\begin{flushright} \large
\emph{} \\
u13230795  
\end{flushright}
\end{minipage}
\begin{minipage}{0.4\textwidth}
\begin{flushleft} \large
Nontokozo Hlastwayo
\end{flushleft}
\end{minipage}
\begin{minipage}{0.4\textwidth}
\begin{flushright} \large
\emph{} \\
u14414555
\end{flushright}
\end{minipage}
\begin{minipage}{0.4\textwidth}
\begin{flushleft} \large
Gerome Schutte
\end{flushleft}
\end{minipage}
\begin{minipage}{0.4\textwidth}
\begin{flushright} \large
\emph{} \\
u12031519
\end{flushright}
\end{minipage}
\vfill

{\large \today}
\end{center}
\end{titlepage}
\footnotesize
\normalsize

%
%
%
%
%
%
%	Table of Contents
%

\tableofcontents
\newpage
\setlength{\parindent}{0em}
\section{Introduction}
When presented with fairly large or small numbers we as human beings tend to struggle. One struggles to grasp the sheer magnitude of how much a trillion dollars is for example or the distance light travels in a second.

The aim of OcuViz is to create a platform for the visualization of data in 3D space using virtual reality to provide an immersive and presence depth.

\section{General Information}
\subsection{System Overview}
\subsection{Description}

OcuViz is a platform for the visualisation of data in a virtual 3D space to allow users to conceptualise large numbers more accurately and naturally. Leveraging the power of VR we are able to create awe-inspiring and immersive scenes for users to experience. As well as, allowing users to create their own scenes through either modular CSV files which are interpreted into 3D scenes or a simplified editor without requiring the user having to be a graphics experts.

\subsubsection{System Configuration}

Minimum hardware requirements to run OcuViz
\begin{itemize}
\item VR Headset:\tab Oculus Rift DK2
\item GPU:\tab[2.1cm] GTX 970 or AMD R9 290
\item CPU:\tab[2.2cm] Intel Core i5-4590
\item RAM:\tab[2.1cm] 8GB
\item USB:\tab[2.3cm] 2 USB 3.0 ports
\item OS:\tab[2.6cm] Windows 7 SP1
\end{itemize}

\subsubsection{Installation}
OcuViz Prerequisite
\begin{itemize}
\item Oculus Rift Runtime Setup - \url{https://www3.oculus.com/en-us/setup/}
\end{itemize}

OcuViz will be provide in both portable and installer formats. To install simply unzip the files (portable format) or follow in the installation instructions.

\subsubsection{Getting Started}
OcuViz is straight forward to use. Ensure your Oculus Rift Headset, controller and head tracker are plugged in. Then launch Oculus Store. Ensure that your headset is showing a blue light. If not check Oculus Store to see the issues that exist. Also accept the health warning message displayed by the h Finally simply run OcuViz.exe to use the software. No login is required and would not make sense for this system.

OcuViz aims to be a user friendly experience. To start using OcuViz is a straight forward procedure.

\begin{enumerate}
\item Ensure Oculus Rift headset, head tracker and controller (or keyboard and mouse) is plugged in.
\item Launch Oculus Rift Runtime.
\item Check that Oculus Rift headset is showing a blue light. 
\item Accept the health warning message if one is displayed on the headset.
\item Run OcuViz.exe 
\end{enumerate}




















%
% End of Document
%
\end{document}